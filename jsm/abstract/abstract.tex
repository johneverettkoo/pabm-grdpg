% !TeX program = pdfLaTeX
\documentclass[12pt]{article}
\usepackage{amsmath}
\usepackage{graphicx,psfrag,epsf}
\usepackage{enumerate}
\usepackage{natbib}
\usepackage{textcomp}
\usepackage[hyphens]{url} % not crucial - just used below for the URL
\usepackage{hyperref}
\providecommand{\tightlist}{%
  \setlength{\itemsep}{0pt}\setlength{\parskip}{0pt}}

%\pdfminorversion=4
% NOTE: To produce blinded version, replace "0" with "1" below.
\newcommand{\blind}{0}

% DON'T change margins - should be 1 inch all around.
\addtolength{\oddsidemargin}{-.5in}%
\addtolength{\evensidemargin}{-.5in}%
\addtolength{\textwidth}{1in}%
\addtolength{\textheight}{1.3in}%
\addtolength{\topmargin}{-.8in}%

%% load any required packages here



% Pandoc citation processing


\begin{document}


\def\spacingset#1{\renewcommand{\baselinestretch}%
{#1}\small\normalsize} \spacingset{1}


%%%%%%%%%%%%%%%%%%%%%%%%%%%%%%%%%%%%%%%%%%%%%%%%%%%%%%%%%%%%%%%%%%%%%%%%%%%%%%

\if0\blind
{
  \title{\bf Popularity Adjusted Block Models are Generalized Random Dot
Product Graphs}

  \author{
        John Koo \\
    Department of Statistics, Indiana University\\
     and \\     Minh Tang \\
    Department of Statistics, North Carolina State University\\
     and \\     Michael W. Trosset \\
    Department of Statistics, Indiana University\\
      }
  \maketitle
} \fi

\if1\blind
{
  \bigskip
  \bigskip
  \bigskip
  \begin{center}
    {\LARGE\bf Popularity Adjusted Block Models are Generalized Random
Dot Product Graphs}
  \end{center}
  \medskip
} \fi

\bigskip
\begin{abstract}
We connect two random graph models, the Popularity Adjusted Block Model
(PABM) and the Generalized Random Dot Product Graph (GRDPG),
demonstrating that a PABM is a GRDPG in which communities correspond to
certain mutually orthogonal subspaces of latent vectors. This insight
leads to the construction of new algorithms for community detection and
parameter estimation for the PABM, as well as improve an existing
algorithm that relies on Sparse Subspace Clustering. Using established
asymptotic properties of Adjacency Spectral Embedding for the GRDPG, we
derive asymptotic properties of these algorithms. In particular, we
demonstrate that the absolute number of community detection errors tends
to zero as the number of graph vertices tends to infinity. Simulation
experiments illustrate these properties.
\end{abstract}

\noindent%
{\it Keywords:} network analysis, community detection, sparse subspace
clustering, spectral clustering
\vfill

\newpage
\spacingset{1.45} % DON'T change the spacing!



\bibliographystyle{agsm}
\bibliography{bibliography.bib}

\end{document}
