% Options for packages loaded elsewhere
\PassOptionsToPackage{unicode}{hyperref}
\PassOptionsToPackage{hyphens}{url}
\PassOptionsToPackage{dvipsnames,svgnames*,x11names*}{xcolor}
%
\documentclass[
]{article}
\usepackage{amsmath,amssymb}
\usepackage{lmodern}
\usepackage{ifxetex,ifluatex}
\ifnum 0\ifxetex 1\fi\ifluatex 1\fi=0 % if pdftex
  \usepackage[T1]{fontenc}
  \usepackage[utf8]{inputenc}
  \usepackage{textcomp} % provide euro and other symbols
\else % if luatex or xetex
  \usepackage{unicode-math}
  \defaultfontfeatures{Scale=MatchLowercase}
  \defaultfontfeatures[\rmfamily]{Ligatures=TeX,Scale=1}
\fi
% Use upquote if available, for straight quotes in verbatim environments
\IfFileExists{upquote.sty}{\usepackage{upquote}}{}
\IfFileExists{microtype.sty}{% use microtype if available
  \usepackage[]{microtype}
  \UseMicrotypeSet[protrusion]{basicmath} % disable protrusion for tt fonts
}{}
\makeatletter
\@ifundefined{KOMAClassName}{% if non-KOMA class
  \IfFileExists{parskip.sty}{%
    \usepackage{parskip}
  }{% else
    \setlength{\parindent}{0pt}
    \setlength{\parskip}{6pt plus 2pt minus 1pt}}
}{% if KOMA class
  \KOMAoptions{parskip=half}}
\makeatother
\usepackage{xcolor}
\IfFileExists{xurl.sty}{\usepackage{xurl}}{} % add URL line breaks if available
\IfFileExists{bookmark.sty}{\usepackage{bookmark}}{\usepackage{hyperref}}
\hypersetup{
  pdftitle={Referee Response for JCGS Submission JCGS-21-341},
  colorlinks=true,
  linkcolor=Maroon,
  filecolor=Maroon,
  citecolor=Blue,
  urlcolor=blue,
  pdfcreator={LaTeX via pandoc}}
\urlstyle{same} % disable monospaced font for URLs
\usepackage[margin=1in]{geometry}
\usepackage{graphicx}
\makeatletter
\def\maxwidth{\ifdim\Gin@nat@width>\linewidth\linewidth\else\Gin@nat@width\fi}
\def\maxheight{\ifdim\Gin@nat@height>\textheight\textheight\else\Gin@nat@height\fi}
\makeatother
% Scale images if necessary, so that they will not overflow the page
% margins by default, and it is still possible to overwrite the defaults
% using explicit options in \includegraphics[width, height, ...]{}
\setkeys{Gin}{width=\maxwidth,height=\maxheight,keepaspectratio}
% Set default figure placement to htbp
\makeatletter
\def\fps@figure{htbp}
\makeatother
\setlength{\emergencystretch}{3em} % prevent overfull lines
\providecommand{\tightlist}{%
  \setlength{\itemsep}{0pt}\setlength{\parskip}{0pt}}
\setcounter{secnumdepth}{-\maxdimen} % remove section numbering
\usepackage{float}
\usepackage{mathtools}
\usepackage{natbib}
\usepackage[linesnumbered,ruled,vlined]{algorithm2e}
\setcitestyle{numbers,square,comma}
\usepackage{verbatim}
\usepackage{amsthm}
\usepackage{comment}
\ifluatex
  \usepackage{selnolig}  % disable illegal ligatures
\fi
\usepackage[]{natbib}
\bibliographystyle{plainnat}

\title{Referee Response for JCGS Submission JCGS-21-341}
\author{}
\date{\vspace{-2.5em}}

\begin{document}
\maketitle

We thank the referees for their time in reviewing this paper and
providing valuable comments and suggestions, which have all been taken
into account for this revision. Most of the changes are in the
simulation and application sections. In particular, we add a simulation
study of ``disassortative'' PABMs and clarify the choices made in the
data preprocessing.

The following sections are individual referee comments and our responses
(highlighted in \textcolor{blue}{blue}).

\hypertarget{referee-1-comments}{%
\section{Referee 1 Comments}\label{referee-1-comments}}

\hypertarget{main-comments}{%
\subsection{Main comments}\label{main-comments}}

The interest of this paper is to provide new perspectives on PABM
(theoretical and algorithmic). The theoretical results are sound.
However, the algorithmic results shown in the simulations are not really
impressive since they are done in settings that do not appear very
challenging. Indeed, they are done with fixed numbers of communities
which are quite low (no more than 4) and only with assortative structure
(larger probabilities of connections within a community than between two
communities). Therefore, I think the paper needs a more challenging
simulation study and the question of the choice of the number of
communities needs to be addressed at least with some heuristic that
works in simulation studies if theoretical results are not reachable.
Moreover, some points on the exposition could be made clearer. I have
specific comments below where I ask for some clarifications and detail
how the simulation study could be improved in my opinion.

\textcolor{blue}{
We thank the referee for the feedback. 
The concerns addressed here are answered below. 
In particular, we made considerable effort to address the referees' concerns about the simulation studies. 
}

\hypertarget{specific-comments-and-questions}{%
\subsection{Specific comments and
questions}\label{specific-comments-and-questions}}

\begin{enumerate}
\def\labelenumi{\arabic{enumi}.}
\item
  p4. What is a hollow graph?\\
  \textcolor{blue}{
  We changed the terminology a bit here and throughout the paper and
  specify that the analysis assumes $G$ is an unweighted and undirected
  graph with no self-loops 
  (no edge directly from itself to itself), 
  so $A$ is binary and symmetric with zeros on the diagonal. The phrase
  ``hollow graph'' is no longer used.  }
\item
  p4. \(G\) as a graph is non-oriented or undirected and \(A\) is
  symmetric.\\
  \textcolor{blue}{
  We have made minor edits clarifying the structure of $G$ and $A$. In
  particular we assume throughout the paper that $G$ is undirected and $A$ is
  symmetric. 
  }
\item
  p4. Please define the set \([n]\).\\
  \textcolor{blue}{
  We added $[n] = \{1, 2, ..., n\}$ at the beginning of Section~2.1. 
  }
\item
  p4. Definition 1. Do you consider a probability distribution on
  \(z_1, ..., z_n\)? Please give the range of variation of \((i, j)\)
  when you define \(P_{ij}\).\\
  \textcolor{blue}{
  The range of $(i, j)$ is now included in the definition; in particular
  we wrote in Section 2.1 that ``$1 \leq i < j \leq n$.''
  In regard to the probability distribution on the labels $z_1, ...,
  z_n$, none of our theoretical results assume anything about the distribution
  of the $\{z_i\}$; we do, however, occasionally assume that the $n_k$
  are sufficiently large for each $k = 1,2,\dots, K$. More specifically 
  the convergence rate of Theorem 5 depends on the $n_k$, i.e., for
  convergence we require that $n_k \to \infty$ as $n \rightarrow
  \infty$. In practice, we also require that each $n_k$ is large enough
  so that the latent vectors in the $k$th community span its
  subspace; this is a rather mild requirement, e.g., if the latent vectors in the $k$th community lie in 
  general positions then we only require $n_k
  \geq K$.}
\item
  p4. line 50. Instead of ``increasing values of the community
  assignments'', you could write ``reorganized by community
  memberships''.\\
  \textcolor{blue}{
  This has been changed.
  }
\item
  p5. Definition 2. I do not understand the definition of the subset
  \(\mathcal{X}\). Is it \(x \in \mathcal{X}\) if for all
  \(y \in \mathbb{R}^d\) we have \(x^\top I_{p,q} y\) or is
  \(\mathcal{X}\) a subset on \((\mathbb{R}^d)^2\)? Please clearly state
  that \(d = p + q\) (not stated before Definition 3).\\
  \textcolor{blue}{
  We now denote $d = p + q$ in Definition 2 and define $\mathcal{X}$ as a subset of $\mathbb{R}^{d}$ such
  that for any $x \in \mathcal{X}$ and $y \in \mathcal{X}$, we have
  $x^{\top} I_{p,q} y \in [0,1]$. One simple way to construct such a set
  $\mathcal{X}$ is to first find a collection of $K$ vectors
  $\mathcal{S} = \{\nu_1, \nu_2, \dots, \nu_K\} \subset \mathbb{R}^{d}$
  where $\nu_i^{\top} I_{p,q} \nu_j \in [0,1]$ for all $i,j$ and then define $\mathcal{X}$
  as 
  $$\mathcal{X} = \mathrm{conv}(\mathcal{S}) = \{x = \sum_{i} \lambda_i \nu_i \colon \lambda_i \geq 0 \,\, \text{for all $i$}, \sum_{i} \lambda_i =
  1\}.$$
  }
\item
  p6. Remark 2. What are \(\hat{Z}\), \(V\), and \(D\)?\\
  \textcolor{blue}{
  We have rewritten Remark~2 to more clearly describe the matrices
  $\hat{Z}$, $V$ and $D$. To avoid confusion, we now use $\hat{D}$ to
  denote the diagonal matrix whose diagonal entries are the $d = p + q$
  largest eigenvalues (in modulus) of $A$ and we use $\hat{V}$ to denote
  the $n \times d$ matrix whose columns are the corresponding
  eigenvectors. We then use $\hat{Z} = \hat{V} |\hat{D}|^{1/2}$ (where
  the $|\cdot|$ operation is applied elementwise) to denote the
  adjacency spectral embedding (ASE) of $A$ into $\mathbb{R}^{d}$; that is
  to say, the $i^{th}$ row of $\hat{Z}$ represents an estimate of the
  latent position $X_i$ (up to some non-identifiability transformation
  $Q \in \mathbb{O}(p,q)$). We also reorganized the paper slightly so
  that the definition and remarks about the non-identifiability of the
  latent positions $X$ in a GRDPG are presented before discussing ASE. }
\item
  top of p7. \(\tilde{P}\) instead of \({P}\).\\
  \textcolor{blue}{
  We have corrected the typo. 
  }
\item
  p7 line 40-46. I don't understand the interest of this paragraph.\\
  \textcolor{blue}{
  The purpose of the additional material in the proof of Theorem 1 is to
  first present the proof in the special case when $K = 2$ so as to
  build intuition for the general case of $K \geq 2$. 
  More specifically for $K = 2$ we show that $\tilde{P} = X \Pi X^\top$
  where $\Pi$ is a permutation matrix with 2 fixed points and 1 cycle of
  order 2, and the latent vectors lie in the union of two $2$-dimensional
  orthogonal subspaces, i.e., the rows of $X$ consist of
  vectors that lie in $\mathcal{S}_1 \cup \mathcal{S}_2$ where
  $\mathcal{S}_1$ and $\mathcal{S}_2$ are both $2$-dimensional
  subspaces of $\mathbb{R}^{4}$ with $\mathcal{S}_1 \cap \mathcal{S}_2 = \{0\}$ and $x^{\top} y
  = 0$ for all $x \in \mathcal{S}_1, y \in \mathcal{S}_2$. 
  This then generalizes to larger $K$ such that the rows of $X$ are now
  vectors that lie in the union of $K$ orthogonal subspaces
  $\mathcal{S}_1 \cup \mathcal{S}_2 \cup \dots \mathcal{S}_K$ where each
  $\mathcal{S}_k$ is a $K$-dimensional subspace of $\mathbb{R}^{K^2}$,
  and $\Pi$ is a permutation matrix with 
  $K$ fixed points and $K (K - 1)$ cycles of order 2. The permutation
  matrix $\Pi$ then has $K(K-1)/2$ eigenvalues equal to $-1$ (due
  to the $K(K-1)$ cycles of order $2$) and $K(K+1)/2$ eigenvalues equal
  to $1$ (due to the $K(K-1)$ cycles of order $2$ together with the $K$
  fixed points). We can thus write $\Pi = U I_{K(K+1)/2,K(K-1)/2}
  U^{\top}$. We have revised the paper to more clearly present the above
  observations. See also Example~1 on page 9 of the revised manuscript
  and the response to comment $10$ below.
  }
\item
  p8, line 14-21. I don't see why the permutation given by \(\Pi\) has
  \(K\) fixed points.\\
  \textcolor{blue}{
  Example 1 illustrates these fixed points explicitly for $K = 3$. 
  More specifically, comparing the matrices $X$ and $Y$, we see that
  they have three columns in common: the first, fifth, and ninth. 
  The other columns are permuted. The proof of Theorem 1 then
  generalizes this to arbitrary $K$. In particular by the construction
  of $X$, each column of $X$ has exactly one non-zero block
  corresponding to some vector $\lambda_{k \ell} \in \mathbb{R}^{n_k}$. The fixed
  points of $\Pi$ are then the columns with indices $r(K+1) +1 $ for $0 \leq r
  \leq K-1$; these columns contain the vectors $\lambda_{kk}$ for
  $1 \leq k \leq K$. The cycles of order $2$ swap the columns of $X$
  containing the vectors $\lambda_{k \ell}$ and $\lambda_{\ell k}$ for
  the $\tbinom{K}{2}$ pairs $\{k,\ell\}$ with $k \not = \ell$. 
  }
\item
  Section 3.1. I found it confusing that community detection algorithms
  are detailed when \(P\) is assumed to be known and then methods for
  estimating \(P\) are recalled. A sentence in the beginning of this
  section to expose what is presented could help the reader.\\
  \textcolor{blue}{
  We thank the referee for pointing out this issue. We had revised the
  manuscript to include the line ``In our methods, the data that are
  observed is only the adjacency matrix $A \sim
  \mathrm{PABM}(\{\lambda^{(k \ell)}\}_K, \rho_n)$ along with an assumed
  number of communities, $K$. To motivate our methods, we first
  consider community detection and parameter estimation in the case
  where we know the edge probability matrix $P$ beforehand, noting
  that community memberships and popularity parameters are not
  immediately discernible from $P$ itself. After establishing methods
  for community detection and parameter estimation from $P$, we use the
  consistency property of ASE to demonstrate that the same methods work
  for $A$ almost surely as $n \to \infty$.'' Indeed, the assumption of
  known $P$ helps motivate Algorithm $1$ through Algorithm~3 and the
  theoretical results in Theorem~2. Theorem 3 and Theorem 4 then follows
  by leveraging the consistency properties for adjacency spectral embedding
  to show that community recovery and parameters estimation using $A$ is asymptotically equivalent to community recovery and parameter estimation 
  using the true but unknown $P$. 
  }
\item
  p10, line 12. \(c^{(i)}\) is not defined.\\
  \textcolor{blue}{
  $c^{(i)}$ is the $i^{th}$ entry of column vector $c$. 
  This has been added to the paper.
  }
\item
  p10. I didn't understand the exposition of the SSC algorithm.\\
  \textcolor{blue}{
  We have added more details to our description of SSC to
  provide further intuition/motivation behind the
  algorithm. The description of SSC now reads:
  ``The SSC algorithm can be described as follows: 
  Given \(X \in \mathbb{R}^{n \times d}\) with vectors
  \(x_i^\top \in \mathbb{R}^d\) as rows of \(X\), the optimization problem
  \(c_i = \arg\min_{c} \|c\|_1\) subject to \(x_i = X^\top c\) and
  \(c^{(i)} = 0\), where $c^{(i)}$ is the $i^{th}$ entry of $c$, is solved for each \(i \in [n]\). 
  The solutions are collected into matrix
  \(C = \bigl[ c_1 \mid \cdots \mid c_n \bigr]^\top\) to
  construct an affinity matrix \(B = |C| + |C^\top|\). If each \(x_i\) lies
  exactly on one of \(K\) subspaces then \(B\) describes an undirected graph
  consisting of {\em at least} \(K\) disjoint subgraphs, i.e., \(B_{ij} = 0\) if \(x_i, x_j\) lie on different subspaces. 
  The intuition here is that vectors that lie on the same subspace can
  be described as linear combinations of each other, assuming the number
  of vectors in the subspace is greater than the dimensionality of the
  subspace. Thus, for each $c_i$, $c_i^{(j)}$ is zero whenever $x_i$ and $x_j$ belong to different subspaces and may be nonzero otherwise. If \(X\) instead represents points near \(K\) subspaces with some noise, 
  then this property will only hold approximately and 
  a final graph partitioning step may be required 
  (e.g., edge thresholding or spectral clustering).''
  }
\item
  p11, line 31. ``\ldots{} SSC on the spectral embedding of \(A\)''.
  Could you support this assertion by references?\\
  \textcolor{blue}{
  There are no references for this, as it is, to the best of our knowledge, a novel idea. 
  We emphasize that while SSC has been proposed for community detection
  for PABM (see e.g., \citep{noroozi2019estimation}), these work apply SSC
  directly on the rows of $A$ (which are binary vectors in
  $\mathbb{R}^{n}$) rather than on the ASE of $A$ (which are real-valued
  vectors in $\mathbb{R}^{d}$ where $d \ll n$). The motivation behind
  our use of SSC comes  from Theorem 1 wherein we show that the latent
  vectors for the PABM lie on the union of orthogonal subspaces in $\mathbb{R}^{d}$ and hence,
  by the consistency properties of ASE, the rows of the estimated
  eigenvectors $\hat{V}$ also lie close to the union of orthogonal
  subspaces in $\mathbb{R}^{d}$. In contrast, while the rows of $P$ do lie exactly on the
  union of $K$ subspaces in $\mathbb{R}^{n}$, the rows of $A$
  (which are now noisy binary vectors) could be quite far from the
  corresponding rows of $P$ and hence need not lie close to the union
  of any $K$ subspaces in $\mathbb{R}^{n}$ 
  }
\item
  p12, line 13. It could help recall what the \(z_i, z_j\) are.\\
  \textcolor{blue}{
  We have changed $z_i$ to ``community labels''.
  }
\item
  p14. Do you think that the OSC algorithm can retrieve other kinds of
  structure than assortative ones?\\
  \textcolor{blue}{
  The OSC algorithm works for any PABM parameters setting and is thus
  agnostic to the particular structure (such as assortativity vs
  disassortativity). We have revised the manuscript to include an
  additional simulation demonstrating this claim (see Section 4.3). In
  particular Section~4.3 considers a PABM setting wherein
  $\lambda_{kk}$ is a vector in $\mathbb{R}^{n_k}$ whose elements
  are i.i.d. $\mathrm{Beta}(1,2)$ random variables and $\lambda_{k\ell}$
  (for $k \not = \ell$) is a
  vector in $\mathbb{R}^{n_k}$ whose elements are i.i.d. $\mathrm{Beta}(2,1)$
  random variables. This result in an average within-community popularity 
  parameter of $1/3$ and a between-community popularity
  parameter of $2/3$. The results for this simulation setting are
  very similar to those presented in the original manuscript and is also
  consistent with the theoretical results in the paper.
  }\\
  \textcolor{blue}{
  On a related note, it is our view that the distinction between
  assortative and disassortative is not always meaningful for the
  PABM. Indeed, unlike the SBM, in the PABM each vertex is free to have a higher affinity to its
  own community or to other communities. For example, suppose vertex $i$
  belongs to community $1$. Then the average probability of an edge
  between vertex $i$ and another vertex in community $1$ is $n^{-1}
  \rho_n 
  \lambda_{i1} \sum_{j} \lambda_{j1}$ while the average probability of an edge between vertex $i$ and another vertex in community $k \not =
  1$ is $n^{-1} \rho_n \lambda_{ik} \sum_{j} \lambda_{j1}$ and thus, on
  average, vertex $i$ has more affinity with vertices in community $k \not =
  1$ whenever $\lambda_{i1} < \lambda_{ik}$
  }
\item
  Sections 4 and 5. Why do you use either the ARI criterion or the
  misclassification rate to assess the recovery of the nodes
  clustering?\\
  \textcolor{blue}{
  The first two datasets that we analyzed in section 5 had also been analyzed
  in previous papers on the PABM. In particular the
  Leeds Butterfly dataset was analyzed in \citep{noroozi2019estimation}
  and while \citep{noroozi2019estimation} provides enough details
  for us to preprocess the data to match their analysis, they do not
  provide code to run their clustering algorithm (which is based on SSC
  applied to the rows of $A$). Therefore we used their reported
  benchmark values, which are given in terms of ARI. 
  In order to make an apples-to-apples comparison, we also report the ARI for OSC and SSC-ASE here as well. 
  For all other analyses (simulation and real data), we use the
  misclassification error rate/count. 
  }
\item
  Do you consider sparsity in the simulated adjacency matrices?\\
  \textcolor{blue}{
  We have added another simulation in the supplemental materials. 
  In this simulation, we fix $n = 2048$ and $K = 3$ and vary $\rho \in
  (0, 1)$. In particular we see that the performance of all algorithms
  deteriorate significantly as the graphs become sparser. Nevertheless
  it appears that the accuracy of OSC is always comparable and/or better
  than the remaining algorithms for all levels of the sparsity
  considered.
  }
\item
  Why do you limit your simulation to assortative structure? Is it
  possible to see the results with a larger value of \(K\)? To what
  extent does the value of \(K\) impact the computational burden for the
  algorithms?\\
  \textcolor{blue}{
  As we discussed in the response to comment
  $16$, we have added a third simulation to address the issue of disassortativity (see Section
  4.3 of the revised manuscript). The performance of all algorithms
  for the setting in Section~4.3 is qualitatively similar to those
  presented earlier in Section~4.1 and Section~4.2 and is consistent with the theory. 
  }\\
  \textcolor{blue}{
  For the simulation studies, we only considered $K \leq 4$ as larger $K$ will, 
  in general, require larger values of $n$ to achieve comparable estimation accuracy as lower values of $K$. 
  Indeed, consider for example estimating the entries of the edge probabilities matrix $P$. 
  Then the rank of $P$ is equal to $K^2$ and hence, for a given $n$, estimation of $P$ when $K = 4$ is considerably simpler than estimation of $P$ when $K = 6$. 
  We have performed some cursory numerical experiments that suggest that OSC does behave similarly to the $K \leq 4$ cases, given sufficient $n$, which is consistent with the theory. 
  }\\
  \textcolor{blue}{
  As for the computational complexity of the algorithms, 
  the main bottleneck for OSC and
  SSC-ASE is the construction of the affinity matrix $B$. 
  For OSC, this involves a (truncated) spectral decomposition to extract the
  $K^2$ largest eigenvalues in modulus (and their corresponding
  eigenvectors) followed by a matrix multiplication of $n \times K^2$
  and $K^2 \times n$ matrices, so the time complexity is of the order $O(n^2 K^2)$. 
  For SSC-ASE (applying SSC on the rows of the adjacency spectral
  embedding $\hat{V}$) we again start with spectral decomposition, followed by
  $n$ LASSO problems with a design matrix of size 
  $K^2 \times (n - 1)$, so the complexity is $O(n^2 K^2 + n^2 K^4 + n K^6)$ \cite{10.1214/009053604000000067}. 
  SSC-A (applying SSC on the rows of $A$) as proposed in \cite{noroozi2019estimation} involves solving $n$ LASSO regression problems each with design matrices of size $n \times (n - 1)$, so the complexity here is $O(n^4)$. 
  In practice the choice of parameter $K$ does not affect
  runtimes too much in our simulation studies, 
  largely because SSC-A takes up the bulk of the runtime. 
  }
\item
  Again, in your application on the Leeds butterfly dataset, why did you
  limit your analysis to \(K = 4\)?\\
  \textcolor{blue}{
  For the Leeds butterfly dataset, we wanted to compare our results
  using OSC and SSC-ASE to an earlier analysis using SSC on the rows of
  $A$ (see \citep{noroozi2019estimation}). In their analysis they
  describe how they removed some of the nodes (corresponding to the
  other butterfly species) so that the resulting dataset only include the
  $K = 4$ butterfly species as considered in our paper.}
\item
  Please specify in the text that the proofs for Theorems 3, 4, and 5
  are given in the Appendix.\\
  \textcolor{blue}{
  We have added this to the end of the introduction section.
  }
\end{enumerate}

\hypertarget{typos}{%
\subsubsection{Typos}\label{typos}}

\begin{enumerate}
\def\labelenumi{\arabic{enumi}.}
\tightlist
\item
  p 10, line 16 ``if each \(x_i\) lieS \ldots{}''\\
  \textcolor{blue}{
  We have corrected this typo. 
  }
\end{enumerate}

\hypertarget{referee-2-comments}{%
\section{Referee 2 Comments}\label{referee-2-comments}}

The first part of the paper shows that any PABM is a GRDPG and gives a
constructive proof of the result. In particular, the proof gives the
construction of the latent positions of the corresponding GRDPG.

The second part of the paper investigates the implications of this
relation to the GRDPG (and the identified latent positions) on
estimation algorithms for the PABM. New algorithms for community
detection and parameter estimation are proposed. In particular, a new
community detection method, named orthogonal spectral clustering (OSC),
is proposed that directly exploits the result of Theorem 1. Furthermore,
an improvement of the sparse subspace clustering (SSC) algorithm is
proposed consisting of applying SSC to the adjacency spectral embedding
of the network (ASE), referred to as SSC-ASE. For both algorithms it is
shown that the subspace detection property holds and that the
convergence rate of the detection error is faster than for classical
SSC. Likewise, it is shown that parameter estimation can be improved. A
numerical study illustrates the performance of the algorithms in
comparison with existing methods. It shows that OSC performs the best,
and also that SSC-ASE improves the results with respect to classical SSC
in some scenarios. In the application on real datasets, OSC and SSC-ASE
provide good results, but do not achieve the performance of modularity
maximization (MM).

The topic of the paper is interesting as it concerns a recent graph
model and the results are valuable. Contributions are two-fold:
theoretic and computational. The paper is well written, and the
presentation of the results and the proofs is very clear.

I only have some minor comments and a couple of questions:

\begin{itemize}
\item
  I have a general issue with latent position models like the GRDPG due
  to the nonidentifiability of the latent space. As stated in the paper,
  latent positions are not unique, and not even the dimension of the
  latent space is identifiable. Theorem 1 (or its proof) provide one of
  all possible latent positions. A general question is what is the
  consequence of this specific choice. Does the latent space (or its
  dimension) have an impact on the clustering results? Indeed, I lack
  intuition about what type of latent space would be optimal for
  community detection or estimation in the PABM. The space with the
  smallest possible dimension? Do the authors have an idea of the
  properties of the latent space that would give the best results for
  community detection? Related to this issue, is the latent space of the
  proof of Theorem 1 the smallest possible latent space to relate PABM
  to the GRDPG?\\
  \textcolor{blue}{
  We thank the referee for the questions. 
  While it is true that there are an infinite number of latent configurations that generate the PABM, based on Theorem 1 and GRDPG results by \citet{rubindelanchy2017statistical}, we can conclusively say the following: 
  \\1. If $P$ is the edge probability matrix for a PABM with $K$
  blocks then $P$ is of rank $K^2$ and furthermore $P$ has $K (K + 1) /
  2$ positive eigenvalues and $K (K - 1) / 2$ negative eigenvalues. 
  \\2. Therefore, viewed as a GRDPG, the latent vectors $X$ can be
  represented as vectors in $\mathbb{R}^{K^2}$ and 
  $K^2$ is the smallest possible dimension. Furthermore the latent
  vectors will always lie on the union of $K$ disjoint $K$-dimensional subspaces---in the configuration outlined by Theorem 1, these subspaces are orthogonal, but the multiplication by arbitrary $Q \in \mathbb{O}(p, q)$ that results in the same $P$ may skew the subspaces making them no longer orthogonal (although they will still be $K$ $K$-dimensional subspaces), 
  \\3. The matrix $B = n V V^\top$, where $V$ is the $n \times K^2$
  matrix of eigenvectors of $P$ (corresponding to the non-zero
  eigenvalues of $P$) is such that $B_{ij} = 0$ if and only
  if vertices $i$ and $j$ are in different communities, i.e.,
  $v_i^{\top} v_j = 0$ if and only if vertices $i$ and $j$ are in
  different communities; here $v_i$ denote the $i^{th}$ row of $V$. 
  }\\
  \textcolor{blue}{
  Therefore, the dimensionality of the latent configuration is not in question, only what kind of effect multiplication by unidentifiable $Q$ has on the configuration. 
  The OSC algorithm circumvents the nonidentifiability of the latent
  configuration by only considering the entries of $\hat{B} = n \hat{V}
  \hat{V}^{\top}$ as $n\hat{V}\hat{V}^{\top}$ is, by the asymptotic properties of
  adjacency spectral embedding, close to $n VV^{\top}$ and $n V
  V^{\top}$ is always unique (the orthogonal projection onto a column
  space is always unique). Similarly, SSC-ASE also circumvents the
  non-identifiability of the latent positions
  by using only the rows of $n^{1/2} \hat{V}$ as two rows $\hat{v}_i$ and
  $\hat{v}_j$ belonging to different communities will 
  have $n \hat{v}_i^{\top} \hat{v}_j \approx 0$ 
  (as Theorem 3 states that $n \hat{v}_i^\top \hat{v}_j \stackrel{a.s.}{\to} 0$). 
  }\\
  \textcolor{blue}{
  That said, if we look at the latent configuration in Theorem 1 then we
  see that a $K$-dimensional (not $K^2$-dimensional) embedding is
  sufficient for clustering, i.e., the embedding corresponding to the
  diagonal blocks $\lambda_{11}, \lambda_{22}, \dots, \lambda_{KK}$. 
  If we are able to find this embedding then this would greatly reduce
  the number of dimensions required for the community
  recovery/clustering. In practice, this isn't always possible due to
  the nonidentifiability issue; in particular, determining which of the
  $K$ out of $K^2$ eigenvectors should be used is the main difficulty.
  Nevertheless we have found via numerical experiments that accurate community
  recovery is generally possible when we use only the eigenvectors corresponding
  to the $2^\mathrm{nd}$ to the $(K+1)^\mathrm{st}$ largest eigenvalues
  of $A$. 
  }
\item
  This is clearly beyond the scope of the paper, but I wonder if there
  is a straightforward way to adapt the proposed methods to directed
  models or weighted graphs or any other types of graphs than binary
  undirected ones?\\
  \textcolor{blue}{
  For directed graphs, the ASE provides a consistent estimator and thus should also work here \citep{doi:10.1080/01621459.2012.699795}. 
  For weighted graphs, if $A_{ij}$ can be an estimator for $P_{ij}$ (e.g., if $A_{ij} \sim \mathrm{Poisson}(P_{ij})$), then all of the theory should transfer over to those models as well, 
  and we believe that the theory can be adapted to other models where $g(A_{ij})$ is an estimator for $P_{ij}$. 
  We hope to address these topics in future research. 
  }
\item
  In the simulation study, SSC-ASE works well for \(K \geq 3\), but not
  for \(K = 2\). The authors explain the bad performance by the use of
  gaussian mixtures in the last clustering step. However, I don't
  understand why this has an impact on the performance for \(K = 2\),
  but not for \(K \geq 3\).\\
  \textcolor{blue}{
  We are not entirely clear on why this happens for $K = 2$, but when
  comparing the normalized Laplacian eigenmaps of the affinity matrix
  $B$ for various $K$, we see that when $K > 2$, the points appear much
  closer to Gaussian mixtures, whereas with $K = 2$, while there is some
  separation by community, they do not appear as Gaussian mixtures. 
  One possible reason for this may be because we failed to find the
  right value for the hyperparameter $\vartheta$ for SSC-ASE. 
  }\\
  \textcolor{blue}{
  More specifically, under ideal circumstances, we can set the hyperparameter $\vartheta$ to
  the theoretical values that ensure, with high probability, the rows of $\hat{V}$ satisfy the subspace
  detection property (SDP). However, as other researchers had pointed out (see
  e.g., \citep{sdp_sufficiency,liu_ssc}, the fact that SDP is satisfied
  does not guarantee that the affinity matrix $B$ represents a graph
  with $K$ connected components but rather a graph with {\em at least}
  $K$ connected component. Indeed, we found in our simulations that
  setting $\vartheta$ to the theoretical values that guarantees 
  SDP often results in $B$ having more than $K$ connected components.
  This is why we choose to do spectral clustering on $B$, i.e., we first
  compute the normalized Laplacian embeddings using $B$ before
  clustering the rows of this embedding using Gaussian mixture models. 
  }\\
  \textcolor{blue}{
  Finally we emphasize that in our simulations the proportion of mislabeled
  vertices does decreases as $n$ increases. For instance, in section 4.1, if $K = 2$ then the average number of mis-clustered vertices for SSC-ASE is around 20 when $n = 128$ and around 100 when
  $n = 4096$; this correspond to an error rate of around $16\%$ and
  $2\%$, respectively. In other words SSC-ASE also works quite well for $K =
  2$ as the proportion of mis-clustered vertices decreases to $0$ as $n$
  increases; it is only when we compared the number of mis-clustered
  vertices that $K = 2$ performs worse compared to $K \geq 3$. 
  }
\item
  In the application section, MM performs better than OSC and SSC-ASE,
  while in the simulation study the opposite is observed. What is the
  reason for the different behavior?\\
  \textcolor{blue}{
  In the application section, we analyze data that are not necessarily generated by the PABM (or any known model). 
  While all three algorithms are specific to the PABM, they might behave
  differently to the different ways in which the model is
  misspecified. Finally, we note that Section~5 analyzed three real
  datasets. For the first dataset (Leeds butterfly) we didn't run the MM
  algorithm as the original implementation of MM used in 
  \cite{307cbeb9b1be48299388437423d94bf1} is extremely slow when $K = 4$. For
  the second dataset (British MP's), the accuracy of MM is slightly
  better than that of OSC. Finally, for the last dataset (Karantaka
  villages), the accuracy of MM is slightly worse than that of OSC. }
\item
  Typo on p.~10: \(X c\) should be \(X^\top c\). Likewise, \(X_{-i} c\)
  is to be replaced with \(X_{-i}^\top c\).\\
  \textcolor{blue}{
  We have corrected the typo.
  }
\end{itemize}

\newpage

  \bibliography{misc.bib}

\end{document}
